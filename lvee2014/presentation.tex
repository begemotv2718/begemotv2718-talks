\input{common}

\title{Нацыянальна свядомы штучны інтэлект}

\begin{document}
\begin{frame}{}
\titlepage
\end{frame}

\begin{frame}{Што маецца на ўвазе}
\begin{center}
{\Large Think IBM Watson}

Машына, якая можа размаўляць з чалавекам і самастойна збіраць інфармацыю з чалавечых тэкстаў
\end{center}
\begin{itemize}
  \item OCR
  \item Аналіз тэкстаў і збор звестак з іх (NLP)
  \item Пазнаванне чалавечай гаворкі 
  \begin{itemize}
    \item TTS
  \end{itemize}
\end{itemize}
  
\end{frame}

\begin{frame}{}
\includegraphics[width=\textwidth]{ai.png}
\end{frame}

\begin{frame}{Рухавікі}
\begin{itemize}
 \item OCR
 \begin{itemize}
  \item cuneiform
  \item tesseract
 \end{itemize}
 \item NLP
 \begin{itemize}
   \item aot?
   \item корпусы
   \item tomita-parser
 \end{itemize}
 \item Гаворка
  \begin{itemize}
   \item CMUSphinx, pocketsphinx
   \item Julius
   \item ...
  \end{itemize}
\end{itemize}
\end{frame}


\begin{frame}{Корпусы тэкстаў}
\begin{itemize}
  \item Ёсць некалькі корпусаў (bnkorpus.info, github.com/poritski/YABC, ...)
  \item Ёсць пытанні з (C)
  \item Ідэі
   \begin{itemize}
     \item crowdsourcing: opencorpora.org engine
     \item Грамадскі набытак: 50 год Шмат класікаў ужо...
   \end{itemize} 
\end{itemize}
\end{frame}

\begin{frame}{Корпусы гаворкі}
  \begin{itemize}
    \item Праект voxforge (nothing/140h)
    \item Праект librivox (1 зборнік вершаў)
  \end{itemize}
\end{frame}

\end{document}
