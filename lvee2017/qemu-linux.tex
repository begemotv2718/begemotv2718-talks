\documentclass[ignorenonframetext, professionalfonts, hyperref={pdftex, unicode}]{beamer}

\usetheme{Belarus}
%\usecolortheme{wolverine}

%Packages to be included
%\usepackage{graphicx}

\usepackage[russian]{babel}
\usepackage[utf8]{inputenc}
\usepackage[T1]{fontenc}

%\usepackage[orientation=landscape, size=custom, width=16, height=9.75, scale=0.5]{beamerposter}

\usepackage{textcomp}

\usepackage{beamerthemesplit}

\usepackage{ulem}

\usepackage{verbatim}

\usepackage{ucs}


\usepackage{listings}
\lstloadlanguages{bash}

\lstset{escapechar=`,
	extendedchars=false,
	language=sh,
	frame=single,
	tabsize=2, 
	columns=fullflexible, 
%	basicstyle=\scriptsize,
	keywordstyle=\color{blue}, 
	commentstyle=\itshape\color{brown},
%	identifierstyle=\ttfamily, 
	stringstyle=\mdseries\color{green}, 
	showstringspaces=false, 
	numbers=left, 
	numberstyle=\tiny, 
	breaklines=true, 
	inputencoding=utf8,
	keepspaces=true,
	morekeywords={u\_short, u\_char, u\_long, in\_addr}
	}

\definecolor{darkgreen}{cmyk}{0.7, 0, 1, 0.5}

\lstdefinelanguage{diff}
{
    morekeywords={+, -},
    sensitive=false,
    morecomment=[l]{//},
    morecomment=[s]{/*}{*/},
    morecomment=[l][\color{darkgreen}]{+},
    morecomment=[l][\color{red}]{-},
    morestring=[b]",
}


\author{Ю.Адамаў, EPAM, Minsk}
\date{23 жніўня 2014}

%\setbeamertemplate{navigation symbols}{}
%\defbeamertemplate*{footline}{decolines theme}{
%  \hbox{
%    \includegraphics[height=0.5cm]{}
%  }
%}


%\institution[EPAM]{EPAM}
%\logo{\includegraphics[width=1cm]{logo.png}}

%\AtBeginSection[]{%
%  \begin{frame}<beamer>
%    \frametitle{}
%    \tableofcontents[
%        sectionstyle=show/shaded, hideallsubsections ]
%  \end{frame}
%  \addtocounter{framenumber}{-1}% If you don't want them to affect the slide number
%}
%
%\AtBeginSubsection[]{%
%  \begin{frame}<beamer>
%    \frametitle{}
%    \tableofcontents[
%        sectionstyle=show/hide,
%        subsectionstyle=show/shaded/hide, ]
%  \end{frame}
%  \addtocounter{framenumber}{-1}% If you don't want them to affect the slide number
%}


\title{Как узнать, что внутри ядра с помощью qemu}

\begin{document}
\begin{frame}{}
\titlepage
\end{frame}

\begin{frame}
  \frametitle{Параметры компиляции}
  \begin{center}
    {\large Подготовка ядра:компиляция с debug символами}
  \end{center}
  \begin{itemize}
    \item Взять исходники ядра с kernel.org
    \item Развернуть исходники \texttt{tar -Jxvf linux-....tar.xz}
    \item В директории исходников \texttt{cp arch/x86/configs/x86\_64\_defconfig ./.config}
    \item \texttt{make menuconfig}
    \item Kernel hacking->Compile-time checks and compiler options->Compile kernel with debug info
    \item \texttt{make}
    \item \texttt { qemu-system-x86\_64 -m 512M -S -gdb tcp::1234 -kernel arch/x86\_64/boot/bzImage -initrd minsystem.image}
    \item \texttt{ gdb ./vmlinux}
  \end{itemize}
\end{frame}


\begin{frame}
  \frametitle{Внутри gdb}
  \begin{itemize}
    \item \texttt{set arch i386:x86-64:intel}
    \item \texttt{target remote localhost:1234}
    \item \texttt{break start\_kernel}
    \item \texttt{continue}
    \item \texttt{disconnect}
    \item \texttt{set arch i386:x86-64}
    \item \texttt{target remote localhost:1234}
    \item Нормально дебажить дальше
  \end{itemize}
\end{frame}

\begin{frame}[fragile]
  \frametitle{Быстрый старт}
\begin{lstlisting}
gdb -ex "set arch i386:x86-64:intel" \
    -ex "target remote localhost:1234" \
    -ex "break start_kernel" \
    -ex "continue" -ex "disconnect" \
    -ex "set arch i386:x86-64" \
    -ex "target remote localhost:1234" ./vmlinux
\end{lstlisting}
\end{frame}
\end{document}
