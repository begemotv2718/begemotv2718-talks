\input{common}

\title{Как узнать, что внутри ядра с помощью qemu}

\begin{document}
\begin{frame}{}
\titlepage
\end{frame}

\begin{frame}
  \frametitle{Параметры компиляции}
  \begin{center}
    {\large Подготовка ядра:компиляция с debug символами}
  \end{center}
  \begin{itemize}
    \item Взять исходники ядра с kernel.org
    \item Развернуть исходники \texttt{tar -Jxvf linux-....tar.xz}
    \item В директории исходников \texttt{cp arch/x86/configs/x86\_64\_defconfig ./.config}
    \item \texttt{make menuconfig}
    \item Kernel hacking->Compile-time checks and compiler options->Compile kernel with debug info
    \item \texttt{make}
    \item \texttt { qemu-system-x86\_64 -m 512M -S -gdb tcp::1234 -kernel arch/x86\_64/boot/bzImage -initrd minsystem.image}
    \item \texttt{ gdb ./vmlinux}
  \end{itemize}
\end{frame}


\begin{frame}
  \frametitle{Внутри gdb}
  \begin{itemize}
    \item \texttt{set arch i386:x86-64:intel}
    \item \texttt{target remote localhost:1234}
    \item \texttt{break start\_kernel}
    \item \texttt{continue}
    \item \texttt{disconnect}
    \item \texttt{set arch i386:x86-64}
    \item \texttt{target remote localhost:1234}
    \item Нормально дебажить дальше
  \end{itemize}
\end{frame}

\begin{frame}[fragile]
  \frametitle{Быстрый старт}
\begin{lstlisting}
gdb -ex "set arch i386:x86-64:intel" \
    -ex "target remote localhost:1234" \
    -ex "break start_kernel" \
    -ex "continue" -ex "disconnect" \
    -ex "set arch i386:x86-64" \
    -ex "target remote localhost:1234" ./vmlinux
\end{lstlisting}
\end{frame}
\end{document}
